IGV
GV is a high-performance visualization tool for interactive
multiple possible views
exploration of large, integrated genomic datasets
It supports a wide variety of data types, including nextgeneration sequence data, and genomic annotations.
One window tool
Allow to move, zoom in and out quickly.
PIXEL RESOLUTION: if data density exceeds the constraint given by the number of pixels available for display, data aggregation is needed before visualization; data aggregation using summary statistics (to be specified by the user)


\begin{figure}[h]
\caption{It is possible to see little portions}
\centering
\includegraphics[width=0.6\textwidth]{}
\label{}
\end{figure}

For each resolution scale (“zoom level”), the aggregated data is divided into tiles that correspond to a region viewable on a typical user display. Each tile is subdivided into bins, with the width of a bin chosen to correspond to the width represented by a pixel at that resolution scale. During the pre-computation step, data in each bin is aggregated into one or more summary statistics as specified by the user. Data Format: the corresponding data tiles for each zoom level are stored in the binary Tiled Data Format, or TDF, which has been optimized for fast tile retrieval. - tile sizes for each zoom level are constant and small, - only the data needed to render the view at the resolution supported by the screen display. - a single tile at the lowest resolution (spanning the entire genome) has the same memory footprint as a tile at the very high zoom levels (might span only a few kilobases). Tiles no longer in view are discarded as needed to free memory. Navigation through a data set is similar to that of Google Maps, allowing the user to zoom and pan seamlessly across the genome at any level of detail from whole genome to base pair.

\begin{figure}[h]
\caption{in correspondance to SNP lower coverage}
\centering
\includegraphics[width=0.6\textwidth]{example}
\label{}
\end{figure}

it is possible to code what we want to see, code xml file and upload it to the software


EXERCISE
Reference genome in fasta file, annotations, and indexed
It is possible to see the single bases

- SNP1
heterozygous SNP

- SNP2
lower coverage
all the cytosines are on the reverse strand, could be associated to tachnical issues